\documentclass[a4paper]{article}

\usepackage[text={18cm, 25cm}, left=1.8cm, top=1.8cm]{geometry}
\usepackage[T1]{fontenc}
\usepackage[utf8x]{inputenc}
\usepackage[linguistics]{forest}
\usepackage[hidelinks,unicode]{hyperref}
\usepackage[czech]{babel}
\usepackage{color}
\usepackage{parskip}

\usepackage{tikz}
\def\checkmark{\tikz\fill[green,scale=0.4](0,.35) -- (.25,0) -- (1,.7) -- (.25,.15) -- cycle;} 

\usepackage{multicol}

\usepackage{amsmath}
\usepackage{amsthm}

\theoremstyle{definition}
\newtheorem{definition}{Definice}

\newcommand{\nter}[1]{\textcolor{blue}{\,#1\,}}
\newcommand{\ter}[1]{\textbf{\textcolor{red}{\,#1\,}}}
\newcommand{\grule}[2]{{\small\nter{#1} $\to$ #2}\\}
\newcommand{\drule}[2]{\checkmark \grule{#1}{#2}}


\title{Gramatická pravidla jazyka IFJ21}

\begin{document}
	\maketitle

	% \section{Definice používané LL gramatiky}

	% \begin{definition}
	% 	Nechť $LL(1) = (V, \Sigma, R, \nter{PROG})$.

	% 	\begin{itemize}
	% 		\item Množina neterminálů $V = \{\nter{PROG}, \nter{EXPR}\}$
			
	% 		\item Množina terminálů (tokenů) $\Sigma = \{\ter{if}, \ter{then}, \ter{do}, \ter{while}, \ter{id}\}$
	% 	\end{itemize}
	% \end{definition}

	\section{Derivační pravidla (množina $R$)}

	\begin{multicols}{2}

	\subsection{Hlavní tělo programu}

	Program vždy musí začínat prologem \texttt{require "ifj21"}.

	\drule{PROG}{\ter{require} \ter{lit\_string} \nter{CODE}}

		Hlavní blok programu (pod \texttt{require "ifj21"}) může obsahovat pouze konstrukce vyhovující pravidlu \nter{TOP\_ELEM} -- tedy pouze deklarace, definice a volání funkcí.

	\drule{CODE}{\ter{$\epsilon$}}
	\drule{CODE}{\nter{TOP\_ELEM} \nter{CODE}}

	\drule{TOP\_ELEM}{\nter{DECL}}
	\grule{TOP\_ELEM}{\nter{DEF}}
	\drule{TOP\_ELEM}{\nter{CALL}}

	Jednotlivé vrchní elementy pak vypadají následovně:

	\drule{DECL}{\ter{global} \ter{id} \ter{:} \ter{function} \ter{(}\nter{TYPE\_LIST}\ter{)} \nter{RET\_LIST}}
	\grule{DEF}{\ter{function} \ter{id} \ter{(} \nter{PARAM\_LIST} \ter{)} \nter{RET\_LIST} \nter{BODY} \ter{end}}
	\drule{CALL}{\ter{id} \ter{(} \nter{ARG\_LIST} \footnote{Ukončovací závorka je v \nter{ARG\_LIST}}} % TODO Toto by melo fungovat jenom pro konstanty jako parametry, volani funkci uvnitr funkci bude resit magic funkce


	\subsection{Typy}

	Seznam typů musí vždy obsahovat alespoň typ \texttt{nil}.

	\drule{RET\_LIST}{\ter{:} \nter{TYPE} \nter{NEXT\_TYPE}}
	\drule{RET\_LIST}{\ter{$\epsilon$}}

	\drule{TYPE\_LIST}{\nter{TYPE} \nter{NEXT\_TYPE}}
	\drule{TYPE\_LIST}{\ter{$\epsilon$}}
	\drule{NEXT\_TYPE}{\ter{,} \nter{TYPE} \nter{NEXT\_TYPE}}
	\drule{NEXT\_TYPE}{\ter{$\epsilon$}}
	\drule{TYPE}{\ter{nil}}
	\drule{TYPE}{\ter{integer}}
	\drule{TYPE}{\ter{number}}
	\drule{TYPE}{\ter{string}}

	\subsection{Argumenty volání funkcí}

	\grule{ARG\_LIST}{\nter{ARG} \nter{NEXT\_ARG}}
	\grule{ARG\_LIST}{\ter{)}}
	\grule{NEXT\_ARG}{\ter{,} \nter{ARG} \nter{NEXT\_ARG}}
	\grule{NEXT\_ARG}{\ter{)}}
	\grule{ARG}{\ter{id}}
	\grule{ARG}{\ter{lit\_number}}
	\grule{ARG}{\ter{lit\_int}}
	\grule{ARG}{\ter{lit\_string}}
	\grule{ARG}{\ter{nil}}

	\subsection{Parametry definice funkcí}

	\grule{PARAM\_LIST}{\nter{PARAM} \nter{NEXT\_PARAM}}
	\grule{NEXT\_PARAM}{\ter{,} \nter{PARAM} \nter{NEXT\_PARAM}}
	\grule{NEXT\_PARAM}{\ter{$\epsilon$}}
	\grule{PARAM}{\ter{id} \ter{:} \nter{TYPE}}

	\subsection{Bloky kódu (sekvence)}

	\grule{BODY}{\ter{$\epsilon$}}
	\grule{BODY}{\nter{VAR\_DECL} \nter{BODY}}
	\grule{BODY}{\nter{IF\_ELSE} \nter{BODY}}
	\grule{BODY}{\nter{WHILE} \nter{BODY}}
	\grule{BODY}{\nter{BODY\_END}}
	\grule{BODY\_END}{\ter{return} \nter{ARG\_LIST}}

	\subsubsection{Deklarace proměnných}
	\grule{VAR\_DECL}{\ter{local} \ter{id} \ter{:} \nter{TYPE} \ter{=} \nter{INIT\_VAL}}
	\subsubsection{Přiřazování proměnných, volání funkcí, \nter{INIT\_VAL}}

	Tyto případy nelze vyjádřit LL(1) gramatikou. Bude se muset použít nějaká zvláštní LL(2) funkce, která se bude dívat dva tokeny dopředu.

	\subsubsection{Řídící konstrukce}

	\grule{IF\_ELSE}{\ter{if} \nter{EXPR} \ter{then} \nter{BODY} \ter{else} \nter{BODY} \ter{end}}
	\grule{WHILE}{\ter{while} \nter{EXPR} \ter{do} \nter{BODY} \ter{end}}

	\end{multicols}

\end{document}
